\usepackage{booktabs}
\usepackage{svg}
\usepackage{tikz}
\usepackage{caption}
\usepackage{subcaption}
\usetikzlibrary{positioning}
\usetikzlibrary{shapes.geometric, arrows}
\tikzstyle{startstop} = [rectangle, rounded corners, minimum width=3cm, minimum height=1cm,text centered, draw=black, fill=red!30]
\tikzstyle{io} = [trapezium, trapezium left angle=70, trapezium right angle=110, minimum width=3cm, text width=3cm, minimum height=1cm, text centered, draw=black, fill=blue!30]
\tikzstyle{process} = [rectangle, minimum width=3cm, minimum height=1cm, text centered, text width=3cm, draw=black, fill=orange!30]
\tikzstyle{decision} = [diamond, minimum width=3cm, minimum height=1cm, text centered, text width=1.5cm, draw=black, fill=green!30]
\tikzstyle{arrow} = [thick,->,>=stealth]

\usepackage{pgfplots}
\usetikzlibrary{pgfplots.groupplots}
\pgfplotsset{compat=1.9,height=0.3\textheight,legend cell align=left,tick scale binop=\times}
\pgfplotsset{grid style={loosely dotted,color=darkgray!30!gray,line width=0.6pt},tick style={black,thin}}
\pgfplotsset{every axis plot/.append style={line width=0.8pt}}

\usepgfplotslibrary{external}
% Für die Verwendung von 'external' müssen die folgenden Anpassungen in Abhängigkeit der
% LaTeX Distribution durchgeführt werden:

% fuer Texlive: pdflatex.exe -shell-escape -synctex=1 -interaction=nonstopmode %.tex
\tikzexternalize[shell escape=-shell-escape]   % fuer TeXLive

% fuer MikTeX:  pdflatex.exe -enable-write18 -synctex=1 -interaction=nonstopmode %.tex
%\tikzexternalize[shell escape=-enable-write18] % fuer MikTex



\tikzsetexternalprefix{graphics/pgfplots/} % Ordner muss ev. zuerst haendisch erstellt werden

\usepackage{enumerate}
\usepackage{color}

\definecolor{mygreen}{rgb}{0,0.6,0}
\definecolor{mygray}{rgb}{0.5,0.5,0.5}
\definecolor{mymauve}{rgb}{0.67,0.69,0.77}

\usepackage{listings}
\lstset{
    frame=tb,
    basicstyle=\fontsize{11}{13}\selectfont\ttfamily,
    captionpos=b,
    tabsize=2,
    numbers=left,
    xleftmargin=2em,
    framexleftmargin=1.5em,
    numberstyle=\color{mygray},
    stringstyle=\color{mygreen},
    morekeywords={with},
    keywordstyle=\bfseries\color{blue},
    abovecaptionskip=0.2em,
    breaklines=true,
    commentstyle=\color{mymauve},
    showstringspaces=false
}

\definecolor{darkgray}{rgb}{.4,.4,.4}
\definecolor{purple}{rgb}{0.65, 0.12, 0.82}
 
%define Javascript language
\lstdefinelanguage{JavaScript}{
    keywords={typeof, new, true, false, catch, function, return, null, catch, switch, var, if, in, while, do, else, case, break},
    keywordstyle=\color{blue}\bfseries,
    ndkeywords={class, export, boolean, throw, implements, import, this, init},
    ndkeywordstyle=\color{darkgray}\bfseries,
    identifierstyle=\color{black},
    sensitive=false,
    comment=[l]{//},
    morecomment=[s]{/*}{*/},
    commentstyle=\color{mymauve}\ttfamily,
    stringstyle=\color{mygreen}\ttfamily,
    morestring=[b]',
    morestring=[b]"
}

    
%define XML language
\definecolor{gray}{rgb}{0.4,0.4,0.4}
\definecolor{darkblue}{rgb}{0.0,0.0,0.6}
\definecolor{cyan}{rgb}{0.0,0.6,0.6}
\lstdefinelanguage{XML}
{
  morestring=[b]",
  morestring=[s]{>}{<},
  morecomment=[s]{<?}{?>},
  stringstyle=\color{black},
  identifierstyle=\color{darkblue},
  keywordstyle=\color{cyan},
  morekeywords={xmlns,version,type}% list your attributes here
}

\usepackage{multirow}
\usepackage{pifont}

\usepackage[nomain,acronym,nonumberlist,nopostdot]{glossaries}
\let\glossarysection\chapter
\renewcommand*{\acronymname}{List of Abbreviations}
\makeglossaries
% \usepackage[xindy]{imakeidx}
% \makeindex