% \glsreset{ros}
\addchap*{Kurzzusammenfassung}
Im Rahmen dieser Diplomarbeit wurde eine grafische Benutzeroberfl\"ache entwickelt, mit Hilfe derer ROS-basierte Roboter programmiert werden k\"onnen. \gls{ros} ist eine Middleware und de facto das Standard-Framework in der Entwicklung von Robotik-Software. Das zweite Framework, das dem entwickeltem Tool zugrunde liegt, ist Blockly - eine von Google entwickelte Open-source JavaScript Bibliothek f\"ur Web-Applikationen, die eine grafische Programmieroberfl\"ache sowie eine Schnittstelle bereitstellt, mit der aus ebenjener grafischen Oberfl\"ache ein ausf\"uhrbarer Code generiert werden kann. Ein Programm besteht damit lediglich aus Bl\"ocken, die zusammengef\"ugt werden. Durch die Verwendung dieser beiden Frameworks ist es m\"oglich, dass das entwickelte Tool auf vielen unterschiedlichen Plattformen eingesetzt werden. \toolname{} - so der Name des Tools, der sich aus den beiden Frameworks ableitet - stellt drei Dienste bereit: eine Oberfl\"ache zum Verwalten der Programmierbl\"ocke, eine Oberfl\"ache zum Erstellen von Programmen und eine Oberfl\"ache zum Verwalten von erstellten Programmen. Aufgrund dieser Eigenschaften bietet das entwickelte Tool L\"osungsans\"atze f\"ur zwei aktuelle Diskussions- und Forschungspunkte im Robotikbereich: das Lernen von Roboterprogrammiermethoden und das Abstrahieren der direkten, mittlerweile sehr komplexen Implementierung auf ein Level, das sehr wenig technische Expertise voraussetzt.
% \glsunset{ros}